\chapter{序論}
\section{研究背景}
移動ロボットの自己位置を推定する方法として確率的な位置推定法である
パーティクルフィルターがよく用いられる。
用いられている例として、rosのナビゲーションスタックである
amclや千葉工業大学未来ロボティクス学科の自律移動ロボットチームが開発した
emclが上げられる。
パーティクルフィルターは現在の状態$(x,y,\theta)$から推定される次のロボットの
状態を多数のパーティクルに見立て、全パーティクルの尤度に基づいた重みつき平均
を次の状態として近似することで自己位置を推定するアルゴリズムである。

安定的に自己位置推定を行うことは、移動ロボットが
ある目標へと自律移動を行うために重要である。
パーティクルフィルターにおいて何らかの理由で
全パーティクルが高い尤度を得られなかった場合、
自己位置を見失ってしまうことがある。
自己位置を見失ったまま行動をし続けてしまった場合、
ロボットが走行できない状態になってしまう。
そういった自己位置を見失ってしまった状態に対しての回復策として
様々なリセット法がある。
例えば、amclは前回より尤度平均が小さくなるとランダムパーティクルを注入をする。
またemclは尤度平均がある任意の閾値より小さくなった場合、
任意の範囲内にパーティクルを配置をする。
それぞれのリセットはいずれも尤度がある値を下回った時に起こり
@@@@な分布を真値へと近づけることができる。

しかし、効果的なリセットではあるが未知障害物を観測した場合、
不要なリセットを何度も行ってしまうことにより逆に自己位置推定に悪影響を与える
ことがある。例えば、既知のマップに対して未知障害物である車や人が観測情報として
入力され観測情報の全てを尤度計算に用いた場合、その分各パーティクルの尤度が小さくなる。
未知障害物への対策が取られていなければ、未知障害物が観測情報として入力されている間は
常に各パーティクルの尤度が小さくなる。そのため、各パーティクルの尤度があるリセットの
閾値より小さい場合はリセットが掛かり続ける。
リセットが掛かりるづけるとパーティクルは収束しないので誘拐されるか自己位置推定を誤ってしまう。

なので安定的に自己位置推定を行うためには未知障害物に対して対応するアルゴリズムが必要である。
未知障害物によって不要なリセットが掛かり続けなければより安定した自己位置推定を行うことができる。

そこで本稿では各パーティクルごとに観測範囲を可変にすることで未知障害物に対して対応する
方法を提案する。この方法は、パーティクルフィルターの性質を利用したものである。
未知障害物が含まれていない観測範囲が与えられたパーティクルであれば、
そのパーティクルの尤度が高くなる。センサ更新を繰り返し行うことで最適な観測範囲が収束し
求められる。またこの方法は、観測範囲を変更するだけなので計算量をさらに低くさせることができ、
よりロバストな自己位置推定を行う事ができる。

この方法については@@@章で話す。

\section{従来研究}

赤井ら・富田ら。。。。

\section{研究目的}

各パーティクルごとに観測範囲を可変にすることで
未知障害物に対して最適な観測範囲を求め不要なリセットを防ぐ。
また未知障害物による尤度の低下を防いだ安定的な自己位置推定のためのアルゴリズムの開発。


\section{本論文の構成}





% 上田\index{うえだ@上田}は、いろいろ書いているが、あまり引用されない。
% 例えば、\cite{上田2015gihyo,ueda2015,上田2015jsai}
% がある。

% \ref{chap:purpose}章で目的を述べる。

% dvipdfmxとhereのテスト
%\begin{figure}[H]
%	\begin{center}
%		\includegraphics[width=1.0\linewidth]{../zero.png}
%		\caption{}
%		\label{fig:}
%	\end{center}
%\end{figure}
%

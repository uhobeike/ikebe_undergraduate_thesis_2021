\chapter{序論}
\section{研究背景}
移動ロボットの自己位置を推定する方法として確率的な位置推定法である
パーティクルフィルターがよく用いられる。
用いられている例として、rosのナビゲーションスタックである
amclが上げられる。
パーティクルフィルターは現在の状態$(x,y,\theta)$から推定される次のロボットの
状態を多数のパーティクルに見立て、全パーティクルの尤度に基づいた重みつき平均
を次の状態として近似することで自己位置を推定するアルゴリズムである。

安定的に自己位置推定を行うことは、移動ロボットが
ある目標へと自律移動を行うために重要である。
パーティクルフィルターにおいて何らかの理由で
各パーティクルが高い尤度を得られなかった場合、
自己位置を見失ってしまうことがある。
自己位置を見失ったまま行動をし続けてしまった場合、
ロボットが走行できない状態になってしまう。
そういった状態の回避策として、自己位置を見失ってしまった状態に対して
様々なリセット法がある。


\section{従来研究}


\section{研究目的}


\section{本論文の構成}





% 上田\index{うえだ@上田}は、いろいろ書いているが、あまり引用されない。
% 例えば、\cite{上田2015gihyo,ueda2015,上田2015jsai}
% がある。

% \ref{chap:purpose}章で目的を述べる。

% dvipdfmxとhereのテスト
%\begin{figure}[H]
%	\begin{center}
%		\includegraphics[width=1.0\linewidth]{../zero.png}
%		\caption{}
%		\label{fig:}
%	\end{center}
%\end{figure}
%

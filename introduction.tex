\chapter{序論}

\section{研究背景}

移動ロボットが自己位置を推定する方法として確率的な位置推定法である
パーティクルフィルターがよく用いられる。
用いられている例として、rosのナビゲーションスタックである
amcl\cite{amcl_github}や千葉工業大学未来ロボティクス学科の上田隆一准教授が
開発したemcl\cite{emcl_github}がある。
%%%%開発者って私じゃなくて?
パーティクルフィルターは現在の状態から推定される次のロボットの
状態の候補として多数のパーティクルに置き換え、
%%%%ちょっと文がおかしい。パーティクルを次のロボットの状態の候補と見立てるんじゃなくて?「見立てる」は別の動詞に変えたほうがいいかも。
予測と観測によりパーティクルの更新を行うことで自己位置を推定するアルゴリズムである。
%%%%重みつき平均じゃなくて最尤なパーティクルの場合もあるのでこれはダウト。もうちょっと大雑把にせつめいしたほうがよい。

安定的に自己位置推定を行うことは、移動ロボットが
ある目標へと自律移動を行うために重要である。
パーティクルフィルターにおいて何らかの理由で
全パーティクルの尤もらしさが低い場合、
自己位置推定を誤ってしまう。
%%%%この時点で尤度とかあんまり使わないほうがいいと思う。もっと誰にでも分かる表現はないでしょうか?
%%%%「見失う」というのは口語です。別の言葉に。
自己位置推定を誤ったまま行動をし続けてしまった場合、
ロボットが走行できない状態になってしまう。
%@@@↑もっと話を膨らます。だとしたらどんな困ったことがあるのか。そして、どうなればよくなるのか。
そのような例として、以下のような例がある。

\begin{itemize}
  \item 観測できない低い障害物と衝突し、スタック
  \item 経路計画が不安定になり、スタック
  \item 誘拐問題\cite{gutmann2002etal}に発展し、スタック
\end{itemize}

自己位置推定を誤ってしまう原因として未知障害物の観測がある。
未知障害物は既知のマップに含まれていない情報のことである。
例えば、既知のマップに対して未知障害物である車や人を含んだ観測情報
をパーティクルの尤度計算に用いた場合、全パーティクルの尤もらしさが低下する。
また、未知障害物への対策を講じていなければ、未知障害物が観測情報に含まれている間は
常に全パーティクルの尤もらしさが低下し続け、パーティクルの分布は大きく広がってしまい、
自己位置推定が破綻してしまう。

自己位置推定を誤ってしまった状態に対しての回復策として
様々なリセット法がある。
例えば、amclは前回より尤度平均が小さくなるとランダムパーティクルを注入する。
emclは尤度平均がある任意の閾値より小さくなった場合、
任意の範囲内にパーティクルを配置をする\cite{ueda2004iros}。
それぞれのリセットはいずれも尤度がある値を下回った時にかかり、
パーティクルの分布を真値周辺に近づけることができる。
しかし、効果的なリセットではあるが未知障害物を観測した場合、
不要なリセットが何度もかかってしまうことで逆に自己位置推定に悪影響を与える
ことがある。そのため、自己位置推定において、未知障害物に対応するための
アルゴリズムが必要である。


%@@@ここで段落は分ける。というか、1.2: 従来研究に移行しましょう。これ以降はまた次の機会に見ます。

\section{従来研究}


%%%%掛かる<-漢字の間違い

% 赤井ら・富田ら。。。。

\section{研究目的}

安定的に自己位置推定を行うためには未知障害物に対して対応するアルゴリズムが必要である。
未知障害物によって不要なリセットが、かかり続けなければ、
より安定した自己位置推定を行うことができる。

そこで本稿では各パーティクルごとに観測範囲を可変にすることで
未知障害物に対して対応する方法を提案する。
未知障害物が含まれていない観測範囲が与えられたパーティクルであれば、
そのパーティクルの尤度は高くなる。
尤度計算を繰り返し行うことで最適な観測範囲が収束し
求められる。
この方法は、観測範囲を変更するだけなので計算量を削減することができ、
よりロバストな自己位置推定を行う事ができる。

% 各パーティクルごとに観測範囲を可変にすることで
% 未知障害物に対して最適な観測範囲を求め不要なリセットを防ぐ。
% また未知障害物による尤度の低下を防いだ安定的な自己位置推定のためのアルゴリズムの開発。

\section{本論文の構成}
この方法については@@@章で話す。

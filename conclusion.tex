\chapter{結論}

本論文では、未知障害物による自己位置推定の不要なリセットを防ぐために、
各パーティクルごとに観測範囲を付与する手法を提案し、アルゴリズムの実装を行った。
また、提案手法が実装された自己位置推定を、あえて未知障害物が存在するシミュレータ環境と実環境
にて使用し、実験を行った。
実験では、あるスタートから目標地点まで自律走行をする際の「走行結果」、「尤度」、
「尤度計算に掛かった時間」についてそれぞれ評価を行った。

結果として、実験結果から、提案手法の実装を行うことによって、未知障害物が存在する環境を走行した場合において
尤度が著しく低下することなく自己位置推定を行うことができることがわかった。
実装が無い場合では未知障害物付近の走行時の尤度が0.5だったのに対して、
実装ありの場合の走行時の尤度は0.8を保つことができた。
また、実装前と比較して、尤度計算に掛かかる時間をシミュレータ環境では4[ms]、実環境では0.4[ms]短縮
することができた。
しかし、実機環境での実験結果においては、障害物が無い箇所の走行時において尤度が
他の時と比べて低下しまっていた。
これは、観測パターンが少ないことや、使用した2D LiDARの捜査角度が270°であったことが原因だと考えられる。
観測パターンを増やすことで、色々な未知障害物の配置において、対応できると考える。
また、捜査角度が360°の2D LiDARを用いることで、より観測パターンを増やすことができることが考えられる。

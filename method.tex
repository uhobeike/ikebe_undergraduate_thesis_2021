\chapter{尤度場モデルを使用したパーティクルフィルタ}
\section{はじめに}
本章では, 尤度場モデルを使用したパーティクルフィルタについて説明をする. 
尤度場モデルを使用したパーティクルフィルタは, 手持ちの地図から事前に求めた, 
マップ上の障害物検知の尤度を表すxy座標上の関数をパーティクルの尤度計算に使用して
自己位置を推定するアルゴリズムである. 
この方法は, 得られるパーティクルの確率分布が雑然とした環境でも滑らかになることや尤度場の事前計算によって
計算効率の高いことがメリットとしてある. 

\section{パーティクルフィルタによる自己位置推定の手順}
\subsection{パーティクルフィルタの初期化}
\subsection{動作モデルによる予測}
\subsection{観測モデルによる尤度計算}
\subsection{リサンプリングによるパーティクルの更新}

\section{課題点}


\chapter{各パーティクルに観測範囲を可変にした未知障害物対策}
\section{はじめに}
この章では, 提案手法である, 各パーティクルに観測範囲を可変にした未知障害物の対策法について
説明する. 

\subsection{パーティクルに観測パターンを付与}
\subsection{尤度計算による観測パターンの評価}
\subsection{リサンプリング時にランダムな観測パターンを付与}
\subsection{尤度計算とリサンプリングによる観測パターンの収束}


